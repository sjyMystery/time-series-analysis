\documentclass[12pt, a4paper]{ctexart}
\usepackage{fancyhdr}
\usepackage{graphicx}
\usepackage{harpoon}
\usepackage{ctex}
\usepackage{amsmath}
\pagestyle{plain}
\usepackage{mathrsfs}
\usepackage{amssymb}

\title{时间序列分析 作业2}
\author{10161511309\,粟嘉逸}
\date{日期 2019/11/26}


\begin{document}
\maketitle{}

\paragraph{题目1}
设$[X,Y]^T$为二维正态向量,求
        \begin{enumerate}
        \item 用$Y$对$X^2$的最小方差估计
        \item 用$Y$对$X^2$的线性无偏最小方差估计
        \end{enumerate}	
\paragraph{解答1}

设$Z=[X,Y]^T$满足正态分布$N(\mu_X,\mu_Y;\sigma_X^2,\sigma_Y^2;\rho)$

1. 用$Y$对$X^2$的最小方差估计

我们已习得,用$Y$对$X^2$的最小方差估计为:$E[X^2|Y]$

我们设$Z$的密度函数为$f(x,y)$,则当$Y=y$时,$X$的条件密度函数为:

\[
    f_{X|Y}(x|y)=\dfrac{f(x,y)}{\int_{-\infty}^{+\infty}f(x,y)dx}
    =\dfrac{1}{\sqrt{2\pi(1-\rho^2)}\sigma_X}exp(-\dfrac{(y-\mu)^2}{2(1-\rho^2)\sigma_X^2})
\]

其中:$\mu_y=\mu_X+(\rho\sigma_X/\sigma_Y)(y-\mu_Y)$

说明已知$Y=y$时,$X$的分布为$N(\mu_y,(1-\rho^2)\sigma_X^2)$

又

\[
    E[X^2|Y=y] = Var[X|Y=y] + (E[X|Y=y])^2
\]

所以

$$
    E[X^2|Y=y] = (1-\rho^2)\sigma_X^2 + \mu_y^2 \\
               = (1-\rho^2)\sigma_X^2 + [\mu_X+(\rho\sigma_X/\sigma_Y)(y-\mu_Y)]^2
$$

从而有

\[
    E[X^2|Y] = (1-\rho^2)\sigma_X^2 + [\mu_X+(\rho\sigma_X/\sigma_Y)(Y-\mu_Y)]^2
\]

2. 用$Y$对$X^2$的线性无偏最小方差估计

我们已习得,该估计表达式为:

\[
    \hat{X^2_Y} = E[X^2] + Cov(X^2,Y)\cdot Cov(Y,Y)^+ (Y-E[Y])
\]

由已知有:

\[
    E[X^2] = Var[X] + E[X]^2 = \sigma_X^2+ \mu_X^2
\]

与
\[
    Y-E[Y]=Y-\mu_Y
\]

特别地,当Y为随机变量的时候,我们不妨将其视作一维的随机向量,于是有:

\[
    Cov(Y,Y)^+ = 1/\sigma_Y^2
\]

根据定义,我们计算得到:

\[
    Cov(X^2,Y)=E[(X^2-E[X^2])(Y-E[Y])]=2\rho\mu_X\sigma_X\sigma_Y
\]

综上有,所求估计式为:

\[
    \hat{X^2_Y} = \sigma_X^2 +\mu_X^2 +2\rho\mu_X\sigma_X/\sigma_Y(Y-\mu_Y)
\]

\paragraph{题目2}
考虑如下的线性随机系统
	\[
	\left\{\begin{array}{rl}
		x(k+1)&=Ax(k)+Dw(k+1)\\
		y(k)&=Cx(k)+Fw(k)
		\end{array}\right.
	\]
	\[
		x(k)\in R^n,y(k)\in R^n,w(k)\in R^n
	\]

	$w(k)$满足$w(k+1)=Mw(k)+\xi(k),M\in R^{m\times m}$,$\xi(k)$为零均值白噪声
    用$y^k$求$x(k),x(k+1)$线性无偏最小方差估计,并求Kalman滤波方程
    
\end{document}





题目二: